\chapter{Overview}
The CBSR Sample Scanner is a Microsoft Windows application that decodes
DataMatrix encoded 1.0 ml or 0.5 ml tubes inserted into NUNC 96 well plates. It
communicates with a scanner that uses either
TWAIN\footnote{\url{http://en.wikipedia.org/wiki/TWAIN}} or
WIA\footnote{\url{http://en.wikipedia.org/wiki/Windows_Image_Acquisition}}
drivers. Most scanners available on the market provide these drivers.

Up to 5 pallets can be scanned and decoded at a time. The product IDs of
decoded pallets can be saved to individual files that specify the pallet
product ID and the date and time the scan was performed. Files are saved in
Comma Separated Value (CSV) format which is a plain text format or can be
imported in to Microsoft Excel.

Figure \ref{fig:main_window} shows the application's main window and highlights
some of its components. This figure shows the application when it is configured
to scan a single pallet.
\begin{figure}[H]
  \centering
  \scalebox{0.35}
	   { \includegraphics*{screenshots/overview/main_window} }
	   \caption{The CBSR Sample Scanner's main window.}
	   \label{fig:main_window}
\end{figure}
\begin{description}
\item[Main Menu] (labelled \emph{1}) Allows the user access to the
  different functions provided by the software (see section \ref{sec:main_menu}
  for a description of the menu items).
\item [Pallet Product IDs] (labelled \emph{2}) The product ID of the
  pallets to be scanned. This product ID is used in the output file. These text
  boxes are enabled or disabled based on which plates are enabled in the
  preferences.
\item [Decoding Profile] (labelled \emph{3}) Scanning profiles control which
  tubes on the pallet will be decoded. Profiles are described in detail in
  section \ref{sec:decoding_profiles}.
\item [Decode Results] (labelled \emph{4}) This part of the window shows the
  decoded product IDs of the tubes that were decoded for each pallet.
\item [Status Bar] (labelled \emph{5}) The status bar is used to display the
  current state of the application. It is usually updated after the user has
  completed a task.
\end{description}

\section{Main Menu}
\label{sec:main_menu}
\subsection{File Menu}
The items under the \emph{File Menu} allow the user to create new scan tabs
and save individual decoded pallets to a file.
\begin{figure}[H]
  \centering
  \scalebox{0.5}
	   { \includegraphics*{screenshots/overview/file_menu} }
	   \caption{File menu.}
	   \label{fig:file_menu}
\end{figure}
\begin{description}
\item[New Scan] Use this menu item to clear the contents of the current
  scanning tabs.
\item[New Tab] Creates a new scanning tab. Multiple scanning tabs can be used
  to keep decode information easily accessible.
\item[Save All] Saves to file all scanned pallets on the current tab.
\item[Save] Used to save to file one of the scanned pallets on the current tab.
\item[Quit] Used to quit the application.
\end{description}

\subsection{Scanner Menu}
Allows the user to save scanned images to file.
\begin{figure}[H]
  \centering
  \scalebox{0.5}
	   { \includegraphics*{screenshots/overview/scanner_menu} }
	   \caption{Scanner menu.}
	   \label{fig:scanner_menu}
\end{figure}
\begin{description}
\item[Scan Image To File] Scans the entire flatbed and saves it to a file in MS
  Windows BMP format.
\item[Scan Pallet to File] Scans the image from a pallet to a file.
\end{description}

\subsection{Configuration Menu}
This menu only contains the preferences item. See Chapter
\ref{chap:configuration} for more details.
\begin{figure}[H]
  \centering
  \scalebox{0.5}
	   { \includegraphics*{screenshots/overview/configuration_menu} }
	   \caption{Configuration menu.}
	   \label{fig:configuration_menu}
\end{figure}

\subsection{Help Menu}
The only item under this menu is the \emph{About} item. When selected it
displays a dialog containing the application's version number.
\begin{figure}[H]
  \centering
  \scalebox{0.5}
	   { \includegraphics*{screenshots/overview/about_dialog} }
	   \caption{About dialog.}
	   \label{fig:about_dialog}
\end{figure}
