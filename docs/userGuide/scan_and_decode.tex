\chapter{Scanning and Decoding}

The first time the application is run a pallet must be configured prior to
scanning and decoding pallets. Please see chapter \ref{chap:configuration} for
instructions on how to configure one or more pallets.
\begin{figure}[H]
  \centering
  \scalebox{0.35}
	   { \includegraphics*{screenshots/scan_and_decode/main_window} }
	   \caption{Application main window.}
	   \label{fig:main_window_2}
\end{figure}
Figure \ref{fig:main_window_2} shows the application when it is configured to
scan a single pallet. When more pallets are configured the bottom part of the
screen will show a decode resutls table for each pallet that is enabled.

To scan one or more pallets follow these steps:
\begin{enumerate}
\item Select the scanning profile to be used. Leave the selection on \emph{All}
  if none have been defined.
\item Enter the pallet barcodes into the pallet barcode text boxes. A USB
  barcode scanner can be used to enter this text into the text boxes.
\item Press the \fbox{Scan Selected} button. Note that only the pallets with
  text in the product ID text box will be scan and decoded.
\item The scanning and decoding progress dialogs will be displayed as shown in
  figure \ref{fig:progress_dialogs}.
\begin{figure}[H]
  \centering
  \scalebox{0.35}
	   { \includegraphics*{screenshots/scan_and_decode/progress_dialogs} }
	   \caption{Progress dialog boxes shown while scanning and decoding a pallet.}
	   \label{fig:progress_dialogs}
\end{figure}
\item When the scan and decode is done the result may be missing one or more
  tubes. In this case press the \fbox{Re-Scan Selected} button to scan and
  decode again. The second scan and decode aggregates information to the
  previous scan and decodes. Figure \ref{fig:scan_missed_tube} shows that the
  tube at position A1 was missed because that cell in the table is
  empty. Re-scan can be attempted multiple times.
\begin{figure}[H]
  \centering
  \scalebox{0.35}
	   { \includegraphics*{screenshots/scan_and_decode/scan_missed_tube} }
	   \caption{Decode with missed tubes.}
	   \label{fig:scan_missed_tube}
\end{figure}
\item Once a pallet is successfully scanned the information can be saved to a
  file using the \texttt{File $\to$ Save All} menu item. If you only want to
  save a single pallet result the use the \texttt{File $\to$ Save $\to$ Pallet
    \emph{x}} menu item. The files are saved with the following name
  \texttt{\textbf{<pallet\_product\_barcode>}\_\textbf{<date\_time>}.csv} to
  the folder of your choice.
\item Once all files have been saved, press the \fbox{Clear Selected} button to
  clear the product ID text boxes and start scanning a new pallet.
\end{enumerate}

\section{Multiple Tabs}
