\chapter{Scanning and Decoding}
This chapter discusses how to use the application for scanning and decoding.

\begin{figure}[H]
  \centering
  \scalebox{0.35}
	   { \includegraphics*{screenshots/scan_and_decode/main_window} }
	   \caption{Application main window.}
	   \label{fig:main_window_2}
\end{figure}
Figure \ref{fig:main_window_2} shows the application when it is configured to
scan a single pallet. When more pallets are configured the bottom part of the
screen will show a decode results table for each pallet that is enabled.

To scan one or more pallets follow these steps:
\begin{enumerate}
\item If this is the first time using the application please configure at least
  one pallet \textbf{plate} (see section \ref{sec:plate_positions}) .
\item Select the scanning profile to be used. Leave the selection on \emph{All}
  if none have been defined.
\item Enter the pallet barcodes into the pallet barcode text boxes. A USB
  barcode scanner can be used to enter this text into the text boxes.
\item Press the \fbox{Scan Selected} button. Note that only the pallets with
  text in the product ID text box will be scan and decoded.
\item The scanning and decoding progress dialogs will be displayed as shown in
  figure \ref{fig:progress_dialogs}.
\begin{figure}[H]
  \centering
  \scalebox{0.35}
	   { \includegraphics*{screenshots/scan_and_decode/progress_dialogs} }
	   \caption{Progress dialog boxes shown while scanning and decoding a pallet.}
	   \label{fig:progress_dialogs}
\end{figure}
\item When the scan and decode is done the result may be missing one or more
  tubes. In this case press the \fbox{Re-Scan Selected} button to scan and
  decode again. The second scan and decode aggregates information to the
  previous scan and decodes. Figure \ref{fig:scan_missed_tube} shows that the
  tube at position A1 was missed because that cell in the table is
  empty. Re-scan can be attempted multiple times.
\begin{figure}[H]
  \centering
  \scalebox{0.35}
	   { \includegraphics*{screenshots/scan_and_decode/scan_missed_tube} }
	   \caption{Decode with missed tubes.}
	   \label{fig:scan_missed_tube}
\end{figure}
\item Once a pallet is successfully scanned the information can be saved to a
  file using the \texttt{File $\to$ Save All} menu item. If you only want to
  save a single pallet result the use the \texttt{File $\to$ Save $\to$ Pallet
    \emph{x}} menu item. The files are saved with the following name
  \texttt{\textbf{<pallet\_product\_barcode>}\_\textbf{<date\_time>}.csv} to
  the folder of your choice.
\item Once all files have been saved, press the \fbox{Clear Selected} button to
  clear the product ID text boxes and start scanning a new pallet.
\end{enumerate}

\section{Error Checking}
To avoid human error, the application remembers the product IDs of the pallets
that have been previously scanned. If it detects a duplicate product ID, the
application will display the dialog box shown in figure
\ref{fig:duplicate_pallet_id}.
\begin{figure}[H]
  \centering
  \scalebox{0.35}
	   { \includegraphics*{screenshots/scan_and_decode/duplicate_pallet_id_dialog} }
	   \caption{Warning dialog displayed when a duplicate pallet ID detected.}
	   \label{fig:duplicate_pallet_id}
\end{figure}
You are given the option of continuing to scan the pallet or cancel the
scan. If more than one plate is enabled, the other pallets will be decoded.

\section{Multiple Tabs}
The application has the ability to display decode information on mutiple
tabs. This may be useful when decoding many pallets and the user wishes to keep
the information displayed on the screen. The user can quickly switch between
the two windows by clicking on their corresponding tabs.
\begin{figure}[H]
  \centering
  \scalebox{0.35}
	   { \includegraphics*{screenshots/scan_and_decode/multiple_tabs} }
	   \caption{Main window with multiple tabs.}
	   \label{fig:multiple_tabs}
\end{figure}
The tab's name displays the date and time that it was created.

Tab's can also be rearranged by clicking on the tab and dragging it. Figure
\ref{fig:split_tabs} shows two tabs split horizontally. This was done by
clicking and dragging the tab down the window until the icon under the mouse
changes to a down arrow.
\begin{figure}[H]
  \centering
  \scalebox{0.35}
	   { \includegraphics*{screenshots/scan_and_decode/split_tabs} }
	   \caption{Split tabs.}
	   \label{fig:split_tabs}
\end{figure}

Note that a tab can be closed at any time by pressing the \textbf{X} on the tab
title. The tab's context menu is shown when you right click on the tab's title.
\begin{figure}[H]
  \centering
  \scalebox{0.35}
	   { \includegraphics*{screenshots/scan_and_decode/tab_context_menu} }
	   \caption{A tab's context menu (right click on the tab title).}
	   \label{fig:tab_context_menu}
\end{figure}

The functions of the items in this menu are as follows:
\begin{description}
\item [Move] Move the tab as to display two tabs side by side either vertically
  or horizontally. This is the same as clicking on the tab and dragging it.
\item [Size] Used for resizing the tab.
\item [Close] Closes the tab.
\item [Close Others] Closes all tabs except this one.
\item [Close All]  Closes all tabs. If this is done a new tab can be created by
  selecting \texttt{File $\to$ New Tab}.
\item [New Editor] Creates a new tab.
\end{description}
